\documentclass{article}

\begin{document}

\section{++a and a++}
    The difference between ++a and a++ is that the result of a++ is a (a changing into a + 1 is an incidental effect),
while the result of the other one is a + 1 (a changing into a + 1 is the main effect). \\
    a-- and --a have the similar effect.

\section{for circulation}
    We can write for circulation as below:
    \begin{verbatim}
    for ( i = number; i < count; i++)
    {
        content
    }
    \end{verbatim}
    Remember, i < count can be replaced with other comparison operators, such as >, <=, etc. ``i++'' is the same.

\section{goto content}
    You can use goto content to switch your model to anywhere you put the word in. For instanse, you can use the content like this:
    \begin{verbatim}
        goto end;
        ...
        end:
    \end{verbatim} 
    Then, your model will come to the end position and continue its operation.

\section{something about char}
    Char variety can represent both some kind of integer, but also string. (\%c represents a char variety.)
    \begin{verbatim}
        char str;

        scanf("%c", str);   //This is to define a char variety, which will soon be translated into an ASCII number.

        operation(str)
        printf("%c", str_new);    //This is to print the str_new string(c represents it), which will be translated from an ASCII number into a string.
        printf("%d", str_new);    //This is to print the ASCII number(d represents it) of str_new string.
    \end{verbatim}

\section{bool}
    In C++, the bool value True represents value 1, while False represents value 0. But we use \%d to output a bool value. In fact, bool value 
is like an integer.

\section{logic operators}
    There are 3 logic operators, or(\verb!||!), and(\&\&), not(!), each of which will output a bool value(0 or 1).
    Remember, the process is that not > and > or.

\section{standarized interval symbols}
    There are three: tab, enter and blank space.

\section{input string}
    When you use scanf to assign a string, the time you use tab or space, it will stop anyway.\par
    But if you want the string to include a space, you can use get(string) sentence.



\section{bitwise operation}
	Bitwise operation is based om the binary codes of integers. It has some algorithms.
	\subsection{\&}
		\begin{table}
			\centering
			\caption{\& algorithm}
			\begin{tabular}{lll}
				result & 0 & 1  \\
				0      & 0 & 0  \\
				1      & 0 & 1 
			\end{tabular}
		\end{table}

\end{document}